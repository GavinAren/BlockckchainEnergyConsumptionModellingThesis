\chapter{Original Project Brief}

\large \textbf{Reason For Change Of Topic}

The primary reason for the topic change was the overly ambitious and unrealistic goal for the project, given the limited resources available.  Most implementations found in the literature were only proposals rather than actual implementations I could improve upon. The learning curve from a programming perspective was also determined to be too steep, given the limited time. 

\LARGE{Vaccine-Tracking Blockchain Solution with Off-Chain Storage for Developing Nations} 

\large \textbf{Problem}

Before a drug is administered to a patient, it changes hands many times in its supply-chain downstream from manufacturers to pharmacies and hospitals. During its manufacturing, storage, distribution, transit and sale, there are many opportunities for it to be stolen, mishandled or even swapped with a counterfeit. These logistical security breaches can lead to dangerous or even deadly outcomes for patients that rely on these drugs. The trade-in counterfeit pharmaceuticals had an 
estimated worth of more than €188 billion in 2018. This especially impacts developing nations like India and low-income African countries. Large amounts of corruption in these countries is yet another hindrance to progress in moving towards a centralised drug-tracking system. Current industry drug-tracking solutions employ IoT and RFID technology based on centralised client-server infrastructures, which come with many problems, especially in the context of developing countries. 
Furthermore, the vaccine supply chain maps out slightly differently to a normal drug’s supply chain due to its sheer volume and its patient ‘Allocation’ stage of its supply chain since it can’t be given to a whole population. A blockchain-based solution would be the perfect solution. There is no single entity mediating and controlling all transactions while making the whole chain transparent to patients and pharmacies. All whilst guaranteeing data integrity through the use of an immutable ledger.

In the USA, under the upcoming DSCSA (Drug Supply Chain Security Act) Act, drug manufacturers will need to ‘provide product tracing information in a secure and interoperable manner electronically to distributors and wholesalers in supply chain’ by November 2023. The big 3 drug wholesalers in the USA are finding this more challenging than expected, and blockchain would once again be the ideal solution.

\large \textbf{Goals}

I want to solve the problems of lack of trust, corruption governments, single points of failure in the network, data integrity and inefficient storage of information on blockchain solutions – through a distributed peer-to-peer network following one immutable ledger as its single source of truth, along with off-chain storage. All to solve the problem of tracking the supply chain of vaccine distribution. I aim to build a (/build upon an existing) blockchain solution for drug tracking and add an off-chain storage solution to it. Very active research area due to the 2023 DSCSA Act deadline incoming. I will be writing smart contracts in languages such as Solidity in conjunction with using Hyperledger Besu in Java to interact with the Ethereum Blockchain, and I also plan to use tools like Truffle. 

\large \textbf{Scope}

Researching problems with drug/vaccine supply chain as well as researching how it specifically impacts low-income developing countries more than others. 

Creating/improving upon an Ethereum-based application that tracks drug supply chain data and adjusting it specifically for the vaccine supply chain adding off-chain storage to it, using IPFS or Swarm. In the end, I will provide metrics on the storage efficiencies as well as the data retrieval speeds from the off-chain part of the solution as compared to other blockchain projects without this feature. As a part of my evaluation, I will be finding the shortcomings of other blockchain projects on the market and find out why they are not being deployed by large drug manufacturers yet. I also plan to evaluate my solution against other private permissioned blockchains, as well as non-blockchain approaches in an unbiased fashion, displaying the various strengths and weaknesses of my project.

