\chapter{Conclusion}

The mathematical model presented in this study accurately estimated the electricity consumption of PoS Ethereum. A thorough exploration of the Ethereum blockchain led to the identification of four novel improvements to a state-of-the-art model. Two of these improvements were incorporated into the model proposed, now accounting for synchronisation energy and operating multiple validators on a single node. The results obtained were satisfactory and fell between those from the only two other models in the field, which were implemented for comparison. Special care was taken to document well-founded justifications, and an innovative methodology was developed that can be applied in future projects. 

The proposed model demonstrated that the switch in the consensus mechanism reduced the electricity consumption of Ethereum by over 99\%. Additionally, research indicated that comparing Visa to blockchain networks is inappropriate, with PayPal being a more suitable representative of centralised banking systems. Results from Model-A reveal that Ethereum's energy consumption per transaction, previously several orders of magnitude greater than PayPal's, has now decreased to just $\sim$31\% higher. This finding is significant as it represents a reasonable trade-off for many users in exchange for the benefits of decentralisation.  

Although this study was based on novel yet robust modelling techniques, it had several limitations detailed in \sref{LimitationsFutureWork} above.


\section{Project Reflection}

Overall, the project goals were achieved. The base model was improved with meaningful additions while still producing valid results.

\subsection{Learnings}

I learnt that it is simple to understand a simplified model of a complex system; however, to be the one modelling it requires absolute subject-matter expertise. Deep knowledge of cryptocurrencies as well as the Ethereum protocol specifically, had to be acquired in order to expand to the pre-existing models meaningfully. 

\subsection{Setbacks}

The original plan of this project was to develop blockchain-based solutions for modelling drug supply chains, but it underwent multiple revisions. The original project brief and the reasons for topic change, can be found in Appendix ****. This was among the foreseen risks in opting for a blockchain-related project as it is a relatively new technology. This risk was mitigated by following the pre-planned risk mitigation strategy found in table**** in the appendix.



Many different approaches to modelling were tried during the course of the project. Some include sensitivity analysis to check which parameters affect the overall electricity consumption the most when using the PoS consensus protocol \cite{MarionAnModelling}. Some data-driven modelling techniques explored included time series analysis and logistic regression \cite{IbanezTheExpansion} however, the 'The Merge' is a very recent event, and it was hard to obtain any meaningful data on it.  ***talk about the risk table

Couldn't use CCRI API as it wasn't ready yet. Has to scrape it from the website from the network section.

Couldn't use API - may have affected the accuracy of data, could've been live data, more precise data


