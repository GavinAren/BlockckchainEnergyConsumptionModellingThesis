\chapter{Conclusion}

The mathematical model presented in this study accurately estimated the electricity consumption of PoS Ethereum. A thorough exploration of the Ethereum blockchain led to the identification of four novel improvements to a state-of-the-art model. Two of these improvements were incorporated into the model proposed, now accounting for synchronisation energy and operating multiple validators on a single node. The results obtained were satisfactory and fell between those from the only two other models in the field, which were implemented for comparison. Special care was taken to document well-founded justifications, and an innovative methodology was developed that can be applied in future projects. 

The proposed model demonstrated that the switch in the consensus mechanism reduced the electricity consumption of Ethereum by over 99\%. Additionally, research indicated that comparing Visa to blockchain networks is inappropriate, with PayPal being a more suitable representative of centralised banking systems. Results from Model-A reveal that Ethereum's energy consumption per transaction, previously several orders of magnitude greater than PayPal's, has now decreased to just $\sim$31\% higher. This finding is significant as it represents a reasonable trade-off for many users in exchange for the benefits of decentralisation.  

Although this study was based on novel yet robust modelling techniques, it had several limitations detailed in \sref{LimitationsFutureWork} above.


\section{Project Reflection}

Overall, the project goals were achieved. The base model was enhanced with meaningful additions while still producing valid results.

\subsection{Learnings}

Many lessons were learned during the course of this study. Understanding a simplified model of a complex decentralised system is straightforward, yet, enhancing such a  model requires subject-matter mastery. Profound knowledge of cryptocurrencies as well as the Ethereum protocol had to be acquired in order to meaningfully expand upon pre-existing models.

\subsection{Setbacks}

Originally, the project was aimed at improving blockchain-based solutions for drug-tracking by adding off-chain storage to them. This underwent multiple revisions, which are explained alongside the outdated project brief found in Appendix ***. This risk was anticipated when choosing a blockchain-related project due to the relatively new nature of the technology. The pre-planned risk mitigation strategy ***, in table ***,  was deployed to tackle this issue.

Different approaches to data-driven modelling were also attempted. Some of these included sensitivity analysis to assess which parameters have the largest impact on the overall electricity consumption when using the PoS consensus protocol \cite{MarionAnModelling}. Time series analysis and logistic regression were also attempted \cite{IbanezTheExpansion}. However, due to the recency of the Ethereum PoS update, obtaining meaningful data proved to be challenging. A similar issue was faced when the CCRI API could not be utilised as it was still in development. This required risk mitigation strategy *** to be followed, and the focus was shifted to modelling with domain knowledge. 


