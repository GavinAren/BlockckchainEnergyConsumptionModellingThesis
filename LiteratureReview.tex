\chapter{Literature Review}
% -_____________________________________________________________________________

\section{Goals And Methodology}


% -_____________________________________________________________________________

\section{Blockchain Basics}




% -_____________________________________________________________________________

\subsection{How it works}





% -_____________________________________________________________________________

\subsection{Consensus in Blockchain Systems}
\url{https://vitalik.ca/general/2017/12/31/pos_faq.}
\url{https://vitalik.ca/general/2017/12/31/pos_faq.html#what-is-weak-subjectivity}


% -_____________________________________________________________________________

\subsection{Types Of Consensus}




% -_____________________________________________________________________________

\subsubsection{Proof-Of-Work}




% -_____________________________________________________________________________

\subsubsection{Proof-Of-Stake}





% -_____________________________________________________________________________

\subsection{Types Of Blockchains}



% -_____________________________________________________________________________

\subsubsection{Public Vs Private Vs Permissioned}


% -_____________________________________________________________________________

\section{Energy Consumption And Carbon Emissions of Blockchain Systems}



% -_____________________________________________________________________________

\subsection{Papers On Carbon Emissions of Blockchain Systems}


% -_____________________________________________________________________________

\subsection{Papers On Modelling Energy Usage Of Blockchain Systems}


% -_____________________________________________________________________________

\subsubsection{Experimental Papers}



% -_____________________________________________________________________________

\subsubsection{Mathematical Papers}


% -_____________________________________________________________________________

\section{Ethereum Basics}


% -_____________________________________________________________________________

\subsection{Pre-Merge Ethereum}

% -_____________________________________________________________________________
\subsection{Post-Merge Ethereum - "The Merge"}

Staking makes the core of 'The Merge', switching from the older Proof of Work to the Proof-of-Stake consensus protocol. Coined as 'The Merge' as the Beacon Chain (a test network) is being merged with the original Ethereum blockchain. Both will operate simultaneously in layers. The older Ethereum chain is now known as the 'execution' layer, and Ethereum is now secured by the Proof-Of-Stake 'consensus' layer (the Beacon Chain). 

This merging of the original chain with the Beacon Chain makes the need for miners redundant. Validator nodes now secure the Ethereum network. 

\paragraph{Validators :}
To participate in the validation of transactions on the Ethereum network, a node must put 32 ETH (equivalent to £45000 currently) at stake. These will be responsible for storing data, processing transactions and adding new blocks to the Ethereum blockchain permanently. 

\paragraph{Staking ETH :}
This can be done in many different ways as not everyone has access to 32 ETH to stake. To get past this high barrier of entry, validator nodes can be operated by solo staking, staking-as-a-service, pooled staking or centralised exchanges.

There are some risks involved with staking. Here are a few:

\textbf{Slashing :}
This is when the algorithm destroys a portion of a validator's stake for behaving maliciously/ against the best interests of the network, introduced as a security feature. These penalties range from 0.5 ETH to the Entire 32 ETH, making it prohibitively expensive to attack Ethereum. 

\textbf{Offline Penalty :}
If a validator goes offline for a number of days, they incur losses roughly equivalent to what they would have gained had they remained online. This is not the same as slashing; instead, the node loses out on the ETH they would have gained during that time.

\textbf{Liquid ETH: }
People operating validator nodes must be prepared to lock up 32 ETH for an indefinite period. There are ways around this, however, through using 3rd parties. 
 
On the other hand, there are many incentives to run validator nodes. An \textbf{Epoch} is a unit of time in Ethereum, usually around 6 minutes. Every epoch, the algorithm judges the validator's actions and dishes out penalties and rewards. These include:

\textbf{Block Proposer Reward :}
This is for those validator nodes chosen randomly to validate the next round of transactions on the blockchain, hence, they are responsible for proposing the next block. When their block is finalised, they are awarded a substantial amount of ETH.



\url{https://twitter.com/VitalikButerin/status/1260314647980322824}



% -_____________________________________________________________________________
\subsection{Ethereum Key Concepts}



% -_____________________________________________________________________________

\section{Key Points Covered}
