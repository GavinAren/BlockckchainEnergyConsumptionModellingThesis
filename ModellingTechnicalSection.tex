\chapter{Modelling The Energy Usage And Carbon Emissions Of Ethereum}
\label{Modelling}

% ____________________________________________________________________________
\section{Chapter Summary}


% ____________________________________________________________________________
\section{Parametric Modelling with Prior Domain Knowledge}

\textbf{Light Nodes}
Light nodes require much less computational power to run compared to a full node, yet, they still work in a trust-minimised way. This is due to their cryptographically proven way of using block headers of the blockchain to verify the information they are receiving from a full 
node themselves. 

Light clients send out a lot of requests, a lot more than full nodes. For example, they might need to check the balance of certain accounts to verify the information they are receiving cryptographically. This large number of simple requests ends up requiring more network bandwidth than full nodes \cite{WhatTechnologies}. We also know that light clients are designed to be run on minimal devices such as mobile and IoT devices. The amount of computational and storage resources required by a light node is orders of magnitude lower than a full node. It requires only about 100MB of storage \cite{WhatTechnologies}. This fact can be used to model this. These devices also sync with the latest blocks using checkpoints in seconds instead of hours like a full node.

\url{https://www.reddit.com/r/ethstaker/comments/x8v3rc/energy_is_expensive_so_how_much_kilowatts_does_a/}


\textbf{Data Used}
From the research,it seems like power usage is usually not the limiting factor but rather the storage space needed to run validators.

The recommended hardware for a validator one is shown in table \_\_ of the appendix as well as in the table below. \_\_

So according to this hardware recommendation, we could estimate the power usage depending on statistics provided by manufacturers. Let's assume the same configuration of hardware as in the \cite{CCRI:Network} (They don't mention type of power supply and mainboard).  According to data from the manufacturers, the power consumption for this configuration comes out to the following: 

\begin{itemize}
    \item Processor TDP - 105W
    
    \item An SSD uses 5W while Active
    
    \item Every 8GB of memory uses 3W, so 16GB of memory will use 6W
    
    \item For a solo validator set-up (non-specialised hardware), miscellaneous components are likely to use another 30W
    
    \item As 92\% efficiency is common, the machine will likely peak at 178W of AC power drawn from the wall
    
    \item The machine will run at idle power consumption 50-90\% of time, so on average it would use around 50-100W
    
\end{itemize}







% ____________________________________________________________________________
\subsection {Key Factors And Limitations}


Energy usage of the defined hardware specification mentioned is denoted by:
\begin{equation*}
    \boldsymbol{\mathrm{E}}
\end{equation*} 

A Consensus Layer Client's energy usage is denoted by subscript:
\begin{equation*}
    \boldsymbol{\mathrm{E}_{CL}}
\end{equation*}

An Execution Layer Client's energy usage is denoted by subscript:
\begin{equation*}
    \boldsymbol{\mathrm{E}_{EL}}
\end{equation*}
 
 An Idle client's energy usage is denoted by: \begin{equation*}
    \boldsymbol{\mathrm{E}_{ID}}
\end{equation*} 

Mean Energy usage for each of the 3 client types are denoted by the following respectively: 
\begin{equation*}
  \boldsymbol{\mathrm{{\overline{E}}_{CL}}}\quad      \boldsymbol{\mathrm{{\overline{E}}_{EL}}}\quad  \boldsymbol{\mathrm{{\overline{E}}_{ID}}}   
\end{equation*}

The number of machines ran with different combinations of consensus and execution layer clients is denoted by:
\begin{equation*}
    \boldsymbol{n}
\end{equation*}

The known share of the network that uses a specific combination of Consensus and Execution layer client is denoted by:
\begin{equation*}
    \boldsymbol{\phi}_{i} \text{ where } {i} \text{ represents a specific CL and EL client combination.}
\end{equation*}

The total share of the network occupied by ${n}$ tested combinations of consensus and execution layer clients is denoted by:
\begin{equation*}
    \boldsymbol{\displaystyle\sum\limits_{i=1}^{n}{\phi_{i}}}
\end{equation*}

\paragraph{First Iteration of the Model}

\begin{equation}
\boldsymbol{\frac{\displaystyle\sum\limits_{i=1}^{n}{ \left({\left(\mathrm{\overline{E}}_{ID} + \mathrm{\overline{E}}_{CL} + \mathrm{\overline{E}}_{EL}\right)} * {\phi_{i}} \right)}}
 {\pi}}
\end{equation}



% ____________________________________________________________________________
\subsection {Modelling }
% ____________________________________________________________________________
\subsection{Ethereum's Total Network Energy Consumption}

% ____________________________________________________________________________
\subsection {Modelling Energy Consumption Per Transaction}

% ____________________________________________________________________________
\subsection{Assumptions}

\begin{itemize}
    \item They have taken the syncing on nodes into account into their model whereas for general modelling, it would be  abetter estimate to ignore this initial set up energy and get an avergae form then on. That is what this paper has done, taking averages of the Raspberry Pi rather than including the syncing energy.
    \item ding ding
\end{itemize}


% _____________________________________________________________________________
\section{ Data-Driven Model}
\subsection{Data Gathering}






% ______________________________________________________________________
\section {Modelling Ethereum's Carbon Emissions}
\url{https://ethernodes.org/countries}

\paragraph{ Simple Linear Regression Model}
A simple logical model for estimating the energy consumption of Ethereum 2.0's energy consumption logically would be using simple regression. y = mx + c, and if we wanted to say that energy is dependent on transactions, then we can say y is the energy per node, m is the slope we need to find, x is the number of transactions/sec, c is the base energy an idle node requires. Then we multiply this whole thing by the number of validators on the network to get the overall network energy consumption.

% ____________________________________________________________________________
\section{Discussion and Evaluation Of Models}
% ____________________________________________________________________________
\subsection{Results}
% ____________________________________________________________________________
\subsection{Interpretation}


% ____________________________________________________________________________
\subsection{Validation and Evaluation}


\tref{Table:tabsubex} shows updates values recorded by CCRI in their report \cite{CCRI:Network} for model \_\_\_\_. %ADD HEREE_________

\begin{table}[!htb]
    \centering

  \subcaptionbox{\textbf{Consensus Clients} \cite{Sigp/blockprint:Metrics}}{
      \begin{tabular}{|l|c|}
            \hline
             Prysm & 65.15 \%  \\
            \hline
             Lighthouse & 29.30 \% \\
            \hline 
             Teku & 10.65 \% \\
            \hline
             Nimbus & 0.91 \% \\
            \hline
             Lodestar & 0.0 \% \\
            \hline
             Other & 0.0 \%  \\
            \hline
  \end{tabular}
    \label{Table:tabsubex:left}
  }
  \subcaptionbox{\textbf{Execution Clients}  \cite{ClientsExplorer}}{
        \begin{tabular}{|l|c|}
            \hline
             Geth & 69.22 \% \\
            \hline
             Nethermind & 14.16 \%  \\
            \hline 
             Erigon & 10.65 \% \\
            \hline
             Besu & 5.78 \% \\
            \hline
             OpenEthereum & 0.00 \% \\
            \hline
             Other & 0.20 \% \\
            \hline
  \end{tabular}
    \label{Table:tabsubex:right}
  }
    \caption{An updated table of client diversity within Ethereum Mainnet network, data recorded on 27-Mar 23. Updates the table from report \cite{CCRI:Network} }
  \label{Table:tabsubex}
\end{table}

Note that the OpenEthereum Execution client has been deprecated and is no longer being maintained. Thus, it is no longer included in the table.

% ____________________________________________________________________________
\subsection{Discussion}
*So I did it this way, could've been done another way. Due to my method, could've affected results
* self reflection -> smaller factors not considered, maybe if I used a differet group of numbers/users some other result would have been achieved, loopholes


Couldn't use API - may have affected the accuracy of data, could've been live data, more precise data

% __________________________________________________________________________



\section{Future Work}
\section{Key Points Covered}
