\chapter{Modelling The Energy Usage And Carbon Emissions Of Ethereum}
\label{Modelling}

% ____________________________________________________________________________
\section{Chapter Summary}


% ____________________________________________________________________________

%  MODEL AAAAAAAAAAAAAAAAAAAAAAAAAAAAAAAAAAAAAAAAAA
\subsection{Model A}
From the research,it seems like power usage is usually not the limiting factor but rather the storage space needed to run validators.

The recommended hardware for a validator one is shown in table \_\_ of the appendix as well as in the table below. \_\_

So according to this hardware recommendation, we could estimate the power usage depending on statistics provided by manufacturers. Let's assume the same configuration of hardware as in the \cite{CCRI:Network} (They don't mention type of power supply and mainboard).  According to data from the manufacturers, the power consumption for this configuration comes out to the following: 

\begin{itemize}
    \item Processor TDP - 105W
    
    \item An SSD uses 5W while Active
    
    \item Every 8GB of memory uses 3W, so 16GB of memory will use 6W
    
    \item For a solo validator set-up (non-specialised hardware), miscellaneous components are likely to use another 30W
    
    \item As 92\% efficiency is common, the machine will likely peak at 178W of AC power drawn from the wall
    
    \item The machine will run at idle power consumption 50-90\% of time, so on average it would use around 50-100W
    
\end{itemize}

% _____________________________________________________________________________
%  MODEL BBBBBBBBBBBBBBBBBBBBBBBBBBBBBBBBBBBB
\section{Model B: Parametric Modelling with Prior Domain Knowledge}

% \textbf{Light Nodes}
% Light nodes require much less computational power to run compared to a full node, yet, they still work in a trust-minimised way. This is due to their cryptographically proven way of using block headers of the blockchain to verify the information they are receiving from a full 
% node themselves. 

% Light clients send out a lot of requests, a lot more than full nodes. For example, they might need to check the balance of certain accounts to verify the information they are receiving cryptographically. This large number of simple requests ends up requiring more network bandwidth than full nodes \cite{WhatTechnologies}. We also know that light clients are designed to be run on minimal devices such as mobile and IoT devices. The amount of computational and storage resources required by a light node is orders of magnitude lower than a full node. It requires only about 100MB of storage \cite{WhatTechnologies}. This fact can be used to model this. These devices also sync with the latest blocks using checkpoints in seconds instead of hours like a full node.

% _____________________---stage 1
\textbf{The CRRI Equation Adapted}
This section introduces the CCRI (\cite{TheNetwork}) equation for the 'weighted  electricity  consumption  of  an  average  node for each combination of one consensus and one execution client'\cite{TheNetwork} with symbols adapted for this paper.

A Consensus Layer Client's power usage is denoted by subscript:
\begin{equation*}
    \boldsymbol{\mathrm{P}_{CL}}
\end{equation*}

An Execution Layer Client's energy usage is denoted by subscript:
\begin{equation*}
    \boldsymbol{\mathrm{P}_{EL}}
\end{equation*}
 
 An Idle client's energy usage is denoted by: \begin{equation*}
    \boldsymbol{\mathrm{P}_{ID}}
\end{equation*} 

Mean Energy usage for each of the 3 client types are denoted by the following respectively: 
\begin{equation*}
  \boldsymbol{\mathrm{{\overline{P}}_{CL}}}\quad      \boldsymbol{\mathrm{{\overline{P}}_{EL}}}\quad  \boldsymbol{\mathrm{{\overline{P}}_{ID}}}   
\end{equation*}

The number of machines ran with different combinations of consensus and execution layer clients is denoted by:
\begin{equation*}
    \boldsymbol{n}
\end{equation*}

The known share of the network that uses a specific combination of Consensus and Execution layer client is denoted by:
\begin{equation*}
    \boldsymbol{\phi}_{EL,CL} \text{ where } {i} \text{ represents a specific CL and EL client combination.}
\end{equation*}

The total share of the network occupied by ${n}$ tested combinations of consensus and execution layer clients is denoted by:
\begin{equation*}
    \boldsymbol{{\pi} = \displaystyle\sum\limits_{i=1}^{n}{\phi_{EL,CL}}}
\end{equation*}

The final equation: 

\begin{equation}
\boldsymbol{\frac{\displaystyle\sum\limits_{i=1}^{n}{ \left({\left(\mathrm{\overline{P}}_{ID} + \mathrm{\overline{P}}_{CL} + \mathrm{\overline{P}}_{EL}\right)} * {\phi_{EL,CL}} \right)}}
 {\pi}}\label{eqn:CCRI}
\end{equation}



% ____________________stage 1
\textbf{Stage 1: Adding the Synchronisation Energy} 

 The equation \ref{eqn:CCRI} ignores the energy expended during the bootstrapping stage of running a node - the synchronisation process. Geth is the most dominant (\tref{Table:tabsubex}) and long-standing Ethereum EL client. Geth's snap sync mode is the most commonly used sync mode as it strikes a great balance between independent verification and sync speed.
% DIAGRAM FOR SNAP SYNC
 All types of syncing modes are very energy-intensive and snap-sync is no exception. It works by downloading the headers of chunks of blocks at a time and verifies these. In parallel to this it starts downloading the state-trie for each block and cryptographically verifying this by regenerating it locally. These computationally intensive processes utilise the CPU to its maximum capacity.

  % keeping Intel i5-1135G7 in mind
The model for calculating the energy consumption of a CPU introduced by this paper \cite{SaingreUnderstandingContracts} is shown below:

\begin{equation*}
    \boldsymbol{\mathrm{P}_{Total} = \mathrm{P}_{idle} + \left({\mathrm{P}_{max} - \mathrm{P}_{idle}}\right) * \mathrm{U}}
\end{equation*}

Where $\boldsymbol{\mathrm{P}_{Total}}$ is the total energy consumption of the CPU, $\boldsymbol{\mathrm{P}_{idle}}$ and $\boldsymbol{\mathrm{P}_{max}}$ denote the power consumption of the CPU in an idle state and under maximum load. $\boldsymbol{\mathrm{U}}$ denotes the percentage of CPU usage under load.

In our case, we want to capture only the energy consumption of the syncing process. For the sake of simplicity, we will assume the $\boldsymbol{\mathrm{U}}$ to be 100\%, or $\boldsymbol{1}$. This negates the need for $\boldsymbol{\mathrm{P}_{idle}}$ to be included in the equation:

\begin{equation*}
    \boldsymbol{\mathrm{P}_{Total} = {\mathrm{P}_{max}}}
\end{equation*}

This paper \cite{Schuchart2016TheScale} shows that the processor cannot continuously operate at maximum capacity for computationally intense applications. To cope, it reduces it's frequency, maintaining operations within the thermal power limitation, the TDP. Hence the TDP can be used as an accurate measurement of the energy consumption of a CPU under a high sustained load. We can infer:

\begin{equation*}
    \boldsymbol{\mathrm{P}_{max} = {\mathrm{P}_{TDP}}}
\end{equation*}

Where the TDP (Thermal Design Power) of the CPU being used by the node is denoted by $\boldsymbol{\mathrm{P}_{TDP}}$, measured in Watts. This can easily be found on the CPU manufacturer's website.

 While the CPU is being utilised at maximum capacity, the node also begins storing this information locally to assemble a local copy of the chain in parallel. This storage process requires speeds that hard drives cannot keep up with. This is reinforced by the recommended specs shown in the \tref{Table:RecommendedHardware} which requires an Solid State Drive with DRAM. These are much faster than traditional hard drives and expend more energy as a consequence. 

 The average energy estimation of an SSD actively in use, in Watts, is denoted by:
 \begin{equation*}
    \boldsymbol{\mathrm{P}_{SSD}} 
\end{equation*}

Often, it takes days to complete a full sync. The time it takes to complete the synchronisation process, in hours, is denoted by:
\begin{equation*}
    \boldsymbol{\mathrm{T}_{SNC}}
\end{equation*}

Thus the total power expended during the synchronisation process $\boldsymbol{\mathrm{P}_{SNC}}$ can be represented as:
\begin{equation}
    \boldsymbol{\mathrm{P}_{SNC} = \mathrm{T}_{SNC} * \left({\mathrm{P}_{TDP}} + \mathrm{P}_{SSD}\right)} \label{eqn:Sync}
\end{equation}

As this is a bootstrapping process, integrating equation \ref{eqn:Sync} into the CCRI equation \ref{eqn:CCRI} results in the following:

\begin{equation*}
    \boldsymbol{\mathrm{P}_{SNC} +  {\frac{\displaystyle\sum\limits_{i=1}^{n}{ \left({\left(\mathrm{\overline{P}}_{ID} + \mathrm{\overline{P}}_{CL} + \mathrm{\overline{P}}_{EL}\right)} * {\phi_{EL,CL}} \right)}}
 {\pi}} } 
\end{equation*}

Which simplifies to:

\begin{equation}
     \boldsymbol{\left({\mathrm{T}_{SNC} * \left({\mathrm{P}_{TDP}} + \mathrm{P}_{SSD}\right)}\right) +  {\frac{\displaystyle\sum\limits_{i=1}^{n}{ \left({\left(\mathrm{\overline{P}}_{ID} + \mathrm{\overline{P}}_{CL} + \mathrm{\overline{P}}_{EL}\right)} * {\phi_{EL,CL}} \right)}}
{\pi}} } \label{eqn:CCRISync}
\end{equation}

\textbf{ Stage 2: Running multiple validators per client}
As the data suggests, there are roughly always 11-15000 nodes online at any given moment ** ** **cite, meanwhile, there are 561,472 validator instances running** ** **cite as of 28 Mar 2023. Most solo-stakers run 1-1000 validator instances per physical node. It is also known that with more specialised hardware, 2500-7000 validators can be run on a single node \url{https://archive.devcon.org/archive/watch/6/validating-made-light-and-simple?tab=YouTube}.






% ___________________________________________________________________MODELS DONE< NOW 

% MODEL BBBBBBBBBBBBBBBBBBBBBBBBBBBBBBBBBBBBBB
\section{Model B Results}

\textbf{Data Gathering}
The recommended hardware configuration for a Geth client is shown in \tref{Table:RecommendedHardware}.

\begin{table}[]
\centering
\begin{tabular}{|l|l|}
\hline
Solo Validator Nodes        & \begin{tabular}[c]{@{}l@{}}Quad Core Processor, \\ 2TB SSD , 16GB memory\end{tabular}                            \\ \hline
Archive Nodes               & \begin{tabular}[c]{@{}l@{}}Quad Core or Dual Core Hyperthreaded \\ Processor, 12TB SSD, 16GB memory\end{tabular} \\ \hline
Minimum for Validator nodes & \begin{tabular}[c]{@{}l@{}}Dual Core Hyperthreaded, \\ 1TB SSD, 4GB memory\end{tabular}                          \\ \hline
\end{tabular}
\caption{Recommended hardware configurations for running various nodes by Geth \cite{2022DeveloperGo-ethereum}}
\label{Table:RecommendedHardware}
\end{table}

A combination of hardware that closely matches the recommendation by Geth developers must be chosen. The following specification was chosen:
Intel i5-1135G7
Because it is a recent CPU, made to be in laptops. This is aligned with the aim of running PoS Ethereum nodes on commodity hardware. It is also a configuration with actual data in the report by \cite{CCRI:Network} and hence will draw fair comparisons when the models are compared.



% ____________________________________________________________________________
\subsection{Ethereum's Total Network Energy Consumption}

% ____________________________________________________________________________
\subsection {Modelling Energy Consumption Per Transaction}

% ____________________________________________________________________________
\subsection{Assumptions}

\begin{itemize}
    \item They have taken the syncing on nodes into account into their model whereas for general modelling, it would be  abetter estimate to ignore this initial set up energy and get an avergae form then on. That is what this paper has done, taking averages of the Raspberry Pi rather than including the syncing energy.
    \item ding ding
\end{itemize}


% _____________________________________________________________________________
\section{ Data-Driven Model}
\subsection{Data Gathering}






% ______________________________________________________________________
\section {Modelling Ethereum's Carbon Emissions}
\url{https://ethernodes.org/countries}

\paragraph{ Simple Linear Regression Model}
A simple logical model for estimating the energy consumption of Ethereum 2.0's energy consumption logically would be using simple regression. y = mx + c, and if we wanted to say that energy is dependent on transactions, then we can say y is the energy per node, m is the slope we need to find, x is the number of transactions/sec, c is the base energy an idle node requires. Then we multiply this whole thing by the number of validators on the network to get the overall network energy consumption.

% ____________________________________________________________________________
\section{Discussion and Evaluation Of Models}
% ____________________________________________________________________________
\subsection{Results}
% ____________________________________________________________________________
\subsection{Interpretation}


% ____________________________________________________________________________
\subsection{Validation and Evaluation}


\tref{Table:tabsubex} shows updates values recorded by CCRI in their report \cite{CCRI:Network} for model \_\_\_\_. %ADD HEREE_________

\begin{table}[!htb]
    \centering

  \subcaptionbox{\textbf{Execution Clients} \cite{Sigp/blockprint:Metrics}}{
      \begin{tabular}{|l|c|}
            \hline
             Geth & 69.22 \% \\
            \hline
             Nethermind & 14.16 \%  \\
            \hline 
             Erigon & 10.65 \% \\
            \hline
             Besu & 5.78 \% \\
            \hline
             OpenEthereum & 0.00 \% \\
            \hline
             Other & 0.20 \% \\
            \hline
  \end{tabular}
    \label{Table:tabsubex:left}
  }
  \subcaptionbox{\textbf{Consensus Clients}  \cite{ClientsExplorer}}{
        \begin{tabular}{|l|c|}
                    \hline
             Prysm & 65.15 \%  \\
            \hline
             Lighthouse & 29.30 \% \\
            \hline 
             Teku & 10.65 \% \\
            \hline
             Nimbus & 0.91 \% \\
            \hline
             Lodestar & 0.0 \% \\
            \hline
             Other & 0.0 \%  \\
            \hline
            
  \end{tabular}
    \label{Table:tabsubex:right}
  }
    \caption{An updated table of client diversity within Ethereum Mainnet network, data recorded on 27-Mar 23. Updates the table from report \cite{CCRI:Network} }
  \label{Table:tabsubex}
\end{table}

Note that the OpenEthereum Execution client has been deprecated and is no longer being maintained. Thus, it is no longer included in the table.

% ____________________________________________________________________________
\subsection{Discussion}
*So I did it this way, could've been done another way. Due to my method, could've affected results
* self reflection -> smaller factors not considered, maybe if I used a differet group of numbers/users some other result would have been achieved, loopholes


Couldn't use API - may have affected the accuracy of data, could've been live data, more precise data

% __________________________________________________________________________



\section{Future Work}
\section{Key Points Covered}
