\chapter{Modelling The Energy Usage And Carbon Emissions Of Ethereum}

% ____________________________________________________________________________
\section{Chapter Summary}


% ____________________________________________________________________________
\section{Parametric Modelling with Prior Domain Knowledge}

% ____________________________________________________________________________
\subsection {Key Factors And Limitations}


Energy usage of the defined hardware specification mentioned is denoted by:
\begin{equation*}
    \boldsymbol{\mathrm{E}}
\end{equation*} 

A Consensus Layer Client's energy usage is denoted by subscript:
\begin{equation*}
    \boldsymbol{\mathrm{E}_{CL}}
\end{equation*}

An Execution Layer Client's energy usage is denoted by subscript:
\begin{equation*}
    \boldsymbol{\mathrm{E}_{EL}}
\end{equation*}
 
 An Idle client's energy usage is denoted by: \begin{equation*}
    \boldsymbol{\mathrm{E}_{ID}}
\end{equation*} 

Mean Energy usage for each of the 3 client types are denoted by the following respectively: 
\begin{equation*}
  \boldsymbol{\mathrm{{\overline{E}}_{CL}}}\quad      \boldsymbol{\mathrm{{\overline{E}}_{EL}}}\quad  \boldsymbol{\mathrm{{\overline{E}}_{ID}}}   
\end{equation*}

The number of machines ran with different combinations of consensus and execution layer clients is denoted by:
\begin{equation*}
    \boldsymbol{n}
\end{equation*}

The known share of the network that uses a specific combination of Consensus and Execution layer client is denoted by:
\begin{equation*}
    \boldsymbol{\phi}_{i} \text{ where } {i} \text{ represents a specific CL and EL client combination.}
\end{equation*}

The total share of the network occupied by ${n}$ tested combinations of consensus and execution layer clients is denoted by:
\begin{equation*}
    \boldsymbol{\displaystyle\sum\limits_{i=1}^{n}{\phi_{i}}}
\end{equation*}

\paragraph{First Iteration of the Model}

\begin{equation}
\boldsymbol{\frac{\displaystyle\sum\limits_{i=1}^{n}{ \left({\left(\mathrm{\overline{E}}_{ID} + \mathrm{\overline{E}}_{CL} + \mathrm{\overline{E}}_{EL}\right)} * {\phi_{i}} \right)}}
 {\pi}}
\end{equation}

\paragraph{ Simple Linear Regression Model}
A simple logical model for estimating the energy consumption of Ethereum 2.0's energy consumption logically would be using simple regression. y = mx + c, and if we wanted to say that energy is dependent on transactions, then we can say y is the energy per node, m is the slope we need to find, x is the number of transactions/sec, c is the base energy an idle node requires. Then we multiply this whole thing by the number of validators on the network to get the overall network energy consumption.



% ____________________________________________________________________________
\subsection {Modelling }
% ____________________________________________________________________________
\subsection{Ethereum's Total Network Energy Consumption}

% ____________________________________________________________________________
\subsection {Modelling Energy Consumption Per Transaction}

% ____________________________________________________________________________
\subsection{Assumptions}

% _____________________________________________________________________________
\section{ Data-Driven Model}
\subsection{Data Gathering}






% ______________________________________________________________________
\section {Modelling Ethereum's Carbon Emissions}
https://ethernodes.org/countries

% ____________________________________________________________________________
\section{Discussion and Evaluation Of Models}
% ____________________________________________________________________________
\subsection{Results}
% ____________________________________________________________________________
\subsection{Interpretation}


% ____________________________________________________________________________
\subsection{Validation and Evaluation}


\tref{Table:tabsubex} shows updates values recorded by CCRI in their report \cite{CCRI:Network} for model \_\_\_\_. %ADD HEREE_________

\begin{table}[!htb]
    \centering

  \subcaptionbox{\textbf{Consensus Clients} \cite{Sigp/blockprint:Metrics}}{
      \begin{tabular}{|l|c|}
            \hline
             Prysm & 65.15 \%  \\
            \hline
             Lighthouse & 29.30 \% \\
            \hline 
             Teku & 10.65 \% \\
            \hline
             Nimbus & 0.91 \% \\
            \hline
             Lodestar & 0.0 \% \\
            \hline
             Other & 0.0 \%  \\
            \hline
  \end{tabular}
    \label{Table:tabsubex:left}
  }
  \subcaptionbox{\textbf{Execution Clients}  \cite{ClientsExplorer}}{
        \begin{tabular}{|l|c|}
            \hline
             Geth & 69.22 \% \\
            \hline
             Nethermind & 14.16 \%  \\
            \hline 
             Erigon & 10.65 \% \\
            \hline
             Besu & 5.78 \% \\
            \hline
             OpenEthereum & 0.00 \% \\
            \hline
             Other & 0.20 \% \\
            \hline
  \end{tabular}
    \label{Table:tabsubex:right}
  }
    \caption{An updated table of client diversity within Ethereum Mainnet network, data recorded on 27-Mar 23. Updates the table from report \cite{CCRI:Network} }
  \label{Table:tabsubex}
\end{table}

Note that the OpenEthereum Execution client has been deprecated and is no longer being maintained. Thus, it is no longer included in the table.

% ____________________________________________________________________________
\subsection{Discussion}
*So I did it this way, could've been done another way. Due to my method, could've affected results
* self reflection -> smaller factors not considered, maybe if I used a differet group of numbers/users some other result would have been achieved, loopholes


Couldn't use API - may have affected accuracy of data, couldv'e been live data, more precise data

% __________________________________________________________________________



\section{Future Work}
\section{Key Points Covered}
