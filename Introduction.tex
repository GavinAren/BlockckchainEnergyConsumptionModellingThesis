\chapter{Introduction}
\url{https://mycelium.xyz/research/environmental-and-societal-impact-of-blockchain}
\section{Motivations}

\subsection{Research Questions}
\label{ResearchQuestions}
\textbf{RQ1} - Using electricity consumption as a proxy of environmental impact, what is the environmental impact of Ethereum 2.0's Proof-of-Stake protocol in comparison to its previous Proof-of-Work protocol and other traditional transactional systems?

\textbf{RQ2} - The Ethereum website claims that this switch to the Proof-of-Stake protocol reduces its electricity consumption by 99.988\%. How accurate is this figure?

\subsection{Motivation}

Over 70\% of all blockchain projects are built on Ethereum (from infura Ethereum page)


\section{Project Goals}
\label{ProjectGoals}


\subsection{Problem}

\subsection{Project Goals}

\subsection{Project Scope}

\section{Novel Contributions}

no one has looked into predicting energy based on the number of validators which is an easily available stat but its behaviour has not been analysed
only 1 other paper focuses on doing exactly this and this paper added 2 important components their equation does not take into account

its a new area, contributed to the data out there, appendix has some extra data others can use and save time