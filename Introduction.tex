\chapter{Introduction}

\section{Problem}
In recent times, cryptocurrencies like Bitcoin have become synonymous with high energy consumption. Bitcoin consumes more energy than large countries like Belgium and has an annual carbon footprint of 55 Mt. This has serious ramifications for the environment \cite{BitcoinDigiconomist}. Bitcoin's biggest competitor, Ethereum, claims to have reduced its energy consumption by 99\% by switching to a Proof-of-Stake (PoS) consensus mechanism, which is far less energy intensive. If a large blockchain like Ethereum, which hosts 70\% of all blockchain projects \cite{EthereumEthereum.org}, can switch to using much less energy-intensive mechanisms, then studying it is extremely important and will help pave the way for other blockchains. 

\subsection{Current State}
While this is a new and topical research area, most studies model Proof-of-Work (PoW) blockchains like Bitcoin, ignoring non-PoW alternatives \cite{Lei2021BestRecommendations}. The occasional models for PoS blockchains that do exist in the literature are outdated and need to incorporate contemporary blockchains, like Ethereum, which went through a major upgrade in Sept 2022, moving away from PoW consensus. 

\section{Goals}
This study aims to create a mathematical model specifically to estimate the energy consumption of PoS Ethereum. This will rely on understanding Ethereum deeply and applying this knowledge to accurately attribute energy consumption to various contributing factors, producing better results than any other model. This model will help answer the research questions posed in \sref{ResearchQuestions}. Additionally, competing models found in the literature will be implemented for comparison.

\subsection{Research Questions}
\label{ResearchQuestions}
\textbf{RQ1} - Using electricity consumption as a proxy of environmental degradation, what is the environmental impact of Ethereum 2.0's PoS mechanism compared to its previous PoW protocol and other traditional centralised transactional systems?

\textbf{RQ2} - Official Ethereum sources claim that this switch to the PoS protocol reduces its electricity consumption by 99.988\% \cite{EthereumEthereum.orgc}. How accurate is this figure?

\section{Scope}
Creating such a model will require an in-depth exploration of all blockchain energy modelling studies and, more specifically, a deep understanding of PoS blockchains. Owing to the novelty of this research area, the limited availability of data considerably constrains the project's scope. Analysing the core concepts of PoS Ethereum and the limited post-Merge data available should reveal which approach and contributing factors will best formulate the model.  

\section{Novel Contributions}
Obtaining accurate estimates for the electricity consumption of cryptocurrencies is a novel topic of research. It can be an especially difficult task due to the decentralised nature of blockchain networks and the cutting-edge nature of the research area. 'Model-A' proposed in this study accounts for factors such as synchronisation energy and operating multiple additional validators on a single node. These factors, amongst others, identified but not included in the model, have been overlooked in past literature as no sufficiently thorough analysis was done on PoS blockchains, especially post-Merge Ethereum.

A comprehensive compilation of data (found in Appendix A and B) was also created through exhaustive research of various contemporary sources. It unveils important behaviours that may contribute to future work in the field.
