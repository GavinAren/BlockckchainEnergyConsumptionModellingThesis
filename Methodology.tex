\chapter {Methodology}
*Have citations to show you methodology is legit

\section {Defining Metrics}


\section {Data Gathering}

Couldn't use the API, had to scrape from the API. Data was annualised for each day's estimate, had to be divided by 365. Could explain why this makes sense sccoriding to their equation, maybe they multiplied by 365, or when they measured energy in the experiment for 2 days, they multiplied by 365/2.

\section {Modelling The Energy Consumption Using Domain Knowledge}

*Define everything in your modelling world, what your definition of everything is

We care about the power coming from the grid. The AC 'At-Wall'. We gather estimations for the recommended configuration of hardware for Ethereum nodes running validator clients. 

Also decide on which way of data gathering is better. Prepare a table of online users claiming their power consumption. Also, estimate the power consumption of the recommended configuration of hardware through manufacturer websites. This is then compared to the data from the \cite{CCRI:Network} report actually running a single validator node to check if this bottom-up hardware estimation approach is valid. 

Knowing the fact that increasing the number of validator clients on a single machine increases the power consumption logarithmically, we apply this assumption to the CCRI equation. We also need to account for power inefficiency of the computer by adding a factor to this equation. The report does not mention the power supply or mainboard used.

Also need to add the syncing energy into equation as it is not a short process. Need to model this, depending on the data and add it to the equation. (possibly for every combination of CL and EL client)


\section {Data-driven Approach To Modelling The Energy Consumption}

\section {Modelling The Carbon Emissions }
* Advice was to make the model, logcally make the equations, justify each assumption etc, then gather some data for it, apply the data see what output you get. Then implement other's models using this data. Finally compare results, say why your model was/was not accurate.

\section {Implementing The Model}
\section {Validating The Model}
\section {Contextualising The Model In A Use Case}

\section {Project Management}
*must be done right here, not later

Couldn't use CCRI API as it wasn't ready yet. Has to scrape it from the website from the network section.